% Options for packages loaded elsewhere
\PassOptionsToPackage{unicode}{hyperref}
\PassOptionsToPackage{hyphens}{url}
%
\documentclass[
  man,floatsintext]{apa6}
\usepackage{amsmath,amssymb}
\usepackage{lmodern}
\usepackage{iftex}
\ifPDFTeX
  \usepackage[T1]{fontenc}
  \usepackage[utf8]{inputenc}
  \usepackage{textcomp} % provide euro and other symbols
\else % if luatex or xetex
  \usepackage{unicode-math}
  \defaultfontfeatures{Scale=MatchLowercase}
  \defaultfontfeatures[\rmfamily]{Ligatures=TeX,Scale=1}
\fi
% Use upquote if available, for straight quotes in verbatim environments
\IfFileExists{upquote.sty}{\usepackage{upquote}}{}
\IfFileExists{microtype.sty}{% use microtype if available
  \usepackage[]{microtype}
  \UseMicrotypeSet[protrusion]{basicmath} % disable protrusion for tt fonts
}{}
\makeatletter
\@ifundefined{KOMAClassName}{% if non-KOMA class
  \IfFileExists{parskip.sty}{%
    \usepackage{parskip}
  }{% else
    \setlength{\parindent}{0pt}
    \setlength{\parskip}{6pt plus 2pt minus 1pt}}
}{% if KOMA class
  \KOMAoptions{parskip=half}}
\makeatother
\usepackage{xcolor}
\usepackage{graphicx}
\makeatletter
\def\maxwidth{\ifdim\Gin@nat@width>\linewidth\linewidth\else\Gin@nat@width\fi}
\def\maxheight{\ifdim\Gin@nat@height>\textheight\textheight\else\Gin@nat@height\fi}
\makeatother
% Scale images if necessary, so that they will not overflow the page
% margins by default, and it is still possible to overwrite the defaults
% using explicit options in \includegraphics[width, height, ...]{}
\setkeys{Gin}{width=\maxwidth,height=\maxheight,keepaspectratio}
% Set default figure placement to htbp
\makeatletter
\def\fps@figure{htbp}
\makeatother
\setlength{\emergencystretch}{3em} % prevent overfull lines
\providecommand{\tightlist}{%
  \setlength{\itemsep}{0pt}\setlength{\parskip}{0pt}}
\setcounter{secnumdepth}{-\maxdimen} % remove section numbering
% Make \paragraph and \subparagraph free-standing
\ifx\paragraph\undefined\else
  \let\oldparagraph\paragraph
  \renewcommand{\paragraph}[1]{\oldparagraph{#1}\mbox{}}
\fi
\ifx\subparagraph\undefined\else
  \let\oldsubparagraph\subparagraph
  \renewcommand{\subparagraph}[1]{\oldsubparagraph{#1}\mbox{}}
\fi
\newlength{\cslhangindent}
\setlength{\cslhangindent}{1.5em}
\newlength{\csllabelwidth}
\setlength{\csllabelwidth}{3em}
\newlength{\cslentryspacingunit} % times entry-spacing
\setlength{\cslentryspacingunit}{\parskip}
\newenvironment{CSLReferences}[2] % #1 hanging-ident, #2 entry spacing
 {% don't indent paragraphs
  \setlength{\parindent}{0pt}
  % turn on hanging indent if param 1 is 1
  \ifodd #1
  \let\oldpar\par
  \def\par{\hangindent=\cslhangindent\oldpar}
  \fi
  % set entry spacing
  \setlength{\parskip}{#2\cslentryspacingunit}
 }%
 {}
\usepackage{calc}
\newcommand{\CSLBlock}[1]{#1\hfill\break}
\newcommand{\CSLLeftMargin}[1]{\parbox[t]{\csllabelwidth}{#1}}
\newcommand{\CSLRightInline}[1]{\parbox[t]{\linewidth - \csllabelwidth}{#1}\break}
\newcommand{\CSLIndent}[1]{\hspace{\cslhangindent}#1}
\ifLuaTeX
\usepackage[bidi=basic]{babel}
\else
\usepackage[bidi=default]{babel}
\fi
\babelprovide[main,import]{english}
% get rid of language-specific shorthands (see #6817):
\let\LanguageShortHands\languageshorthands
\def\languageshorthands#1{}
% Manuscript styling
\usepackage{upgreek}
\captionsetup{font=singlespacing,justification=justified}

% Table formatting
\usepackage{longtable}
\usepackage{lscape}
% \usepackage[counterclockwise]{rotating}   % Landscape page setup for large tables
\usepackage{multirow}		% Table styling
\usepackage{tabularx}		% Control Column width
\usepackage[flushleft]{threeparttable}	% Allows for three part tables with a specified notes section
\usepackage{threeparttablex}            % Lets threeparttable work with longtable

% Create new environments so endfloat can handle them
% \newenvironment{ltable}
%   {\begin{landscape}\centering\begin{threeparttable}}
%   {\end{threeparttable}\end{landscape}}
\newenvironment{lltable}{\begin{landscape}\centering\begin{ThreePartTable}}{\end{ThreePartTable}\end{landscape}}

% Enables adjusting longtable caption width to table width
% Solution found at http://golatex.de/longtable-mit-caption-so-breit-wie-die-tabelle-t15767.html
\makeatletter
\newcommand\LastLTentrywidth{1em}
\newlength\longtablewidth
\setlength{\longtablewidth}{1in}
\newcommand{\getlongtablewidth}{\begingroup \ifcsname LT@\roman{LT@tables}\endcsname \global\longtablewidth=0pt \renewcommand{\LT@entry}[2]{\global\advance\longtablewidth by ##2\relax\gdef\LastLTentrywidth{##2}}\@nameuse{LT@\roman{LT@tables}} \fi \endgroup}

% \setlength{\parindent}{0.5in}
% \setlength{\parskip}{0pt plus 0pt minus 0pt}

% Overwrite redefinition of paragraph and subparagraph by the default LaTeX template
% See https://github.com/crsh/papaja/issues/292
\makeatletter
\renewcommand{\paragraph}{\@startsection{paragraph}{4}{\parindent}%
  {0\baselineskip \@plus 0.2ex \@minus 0.2ex}%
  {-1em}%
  {\normalfont\normalsize\bfseries\itshape\typesectitle}}

\renewcommand{\subparagraph}[1]{\@startsection{subparagraph}{5}{1em}%
  {0\baselineskip \@plus 0.2ex \@minus 0.2ex}%
  {-\z@\relax}%
  {\normalfont\normalsize\itshape\hspace{\parindent}{#1}\textit{\addperi}}{\relax}}
\makeatother

% \usepackage{etoolbox}
\makeatletter
\patchcmd{\HyOrg@maketitle}
  {\section{\normalfont\normalsize\abstractname}}
  {\section*{\normalfont\normalsize\abstractname}}
  {}{\typeout{Failed to patch abstract.}}
\patchcmd{\HyOrg@maketitle}
  {\section{\protect\normalfont{\@title}}}
  {\section*{\protect\normalfont{\@title}}}
  {}{\typeout{Failed to patch title.}}
\makeatother

\usepackage{xpatch}
\makeatletter
\xapptocmd\appendix
  {\xapptocmd\section
    {\addcontentsline{toc}{section}{\appendixname\ifoneappendix\else~\theappendix\fi\\: #1}}
    {}{\InnerPatchFailed}%
  }
{}{\PatchFailed}
\keywords{language development, vocabulary, individual differences, Item Response Models\newline\indent Word count: X}
\usepackage{lineno}

\linenumbers
\usepackage{csquotes}
\ifLuaTeX
  \usepackage{selnolig}  % disable illegal ligatures
\fi
\IfFileExists{bookmark.sty}{\usepackage{bookmark}}{\usepackage{hyperref}}
\IfFileExists{xurl.sty}{\usepackage{xurl}}{} % add URL line breaks if available
\urlstyle{same} % disable monospaced font for URLs
\hypersetup{
  pdftitle={PREVIC: An Item Response Theory based parent report measure of expressive vocabulary in children between 3 and 8 years of age},
  pdfauthor={Manuel Bohn1,2, Julia Prein1, Tobias Koch3, Maximilian Bee3, Daniel Haun2, \& Natalia Gagarina4},
  pdflang={en-EN},
  pdfkeywords={language development, vocabulary, individual differences, Item Response Models},
  hidelinks,
  pdfcreator={LaTeX via pandoc}}

\title{PREVIC: An Item Response Theory based parent report measure of expressive vocabulary in children between 3 and 8 years of age}
\author{Manuel Bohn\textsuperscript{1,2}, Julia Prein\textsuperscript{1}, Tobias Koch\textsuperscript{3}, Maximilian Bee\textsuperscript{3}, Daniel Haun\textsuperscript{2}, \& Natalia Gagarina\textsuperscript{4}}
\date{}


\shorttitle{Parental report measure of vocabulary}

\authornote{

We thank Susanne Mauritz for her help with the data collection.

The authors made the following contributions. Manuel Bohn: Conceptualization, Formal Analysis, Writing - Original Draft Preparation, Writing - Review \& Editing; Julia Prein: Conceptualization, Software, Writing - Original Draft Preparation, Writing - Review \& Editing; Tobias Koch: Formal Analysis, Writing - Review \& Editing; Maximilian Bee: Formal Analysis, Writing - Review \& Editing; Daniel Haun: Conceptualization, Writing - Review \& Editing; Natalia Gagarina: Conceptualization, Writing - Original Draft Preparation, Writing - Review \& Editing.

Correspondence concerning this article should be addressed to Manuel Bohn, Max Planck Institute for Evolutionary Anthropology, Deutscher Platz 6, 04103 Leipzig, Germany. E-mail: \href{mailto:manuel_bohn@eva.mpg.de}{\nolinkurl{manuel\_bohn@eva.mpg.de}}

}

\affiliation{\vspace{0.5cm}\textsuperscript{1} Institute for Psychology, Leuphana University Lüneburg, Germany\\\textsuperscript{2} Department of Comparative Cultural Psychology, Max Planck Institute for Evolutionary Anthropology, Leipzig, Germany\\\textsuperscript{3} Institut of Psychology, Friedrich-Schiller-University Jena, Germany\\\textsuperscript{4} Leibniz-Zentrum Allgemeine Sprachwissenschaft, Berlin, Germany}

\abstract{%
lore ipsum.
}



\begin{document}
\maketitle

\hypertarget{introduction}{%
\section{Introduction}\label{introduction}}

Learning language is one of the key developmental objectives for children. This learning process is highly variable and leads to persistent individual differences which are related to a wide range of outcome measures later in life (Bleses, Makransky, Dale, Højen, \& Ari, 2016; Bornstein, Hahn, Putnick, \& Pearson, 2018; Roberta Michnick Golinkoff, Hoff, Rowe, Tamis-LeMonda, \& Hirsh-Pasek, 2019; Marchman \& Fernald, 2008; Morgan, Farkas, Hillemeier, Hammer, \& Maczuga, 2015; Pace, Alper, Burchinal, Golinkoff, \& Hirsh-Pasek, 2019; Pace, Luo, Hirsh-Pasek, \& Golinkoff, 2017; Schoon, Parsons, Rush, \& Law, 2010; Walker, Greenwood, Hart, \& Carta, 1994). For example, in a longitudinal study spanning 29 years, Schoon et al. (2010) found that relatively poorer language skills at age five were associated with lower levels of mental health at age 34. Given their high predictive validity, high-quality measures are needed to assess early language abilities.

Child language measures can be broadly categorized into two types: direct and parent report measures. Direct measures are generally used with children of three years and older. Direct expressive language assessments involve prompting children to generate words or sentences in response to a given stimulus, such as a picture depicting a scene or an object. Direct receptive language assessments require children to match a verbal prompt with a corresponding picture of a scene or object. Various direct measures tailored to different languages and age groups have been developed, including measures for English and German (Dunn \& Dunn, 1965; Dunn, Dunn, Whetton, \& Burley, 1997; Glück \& Glück, 2011; Roberta M. Golinkoff et al., 2017; Kauschke \& Siegmüller, 2002; Kiese-Himmel, 2005; Lenhard, Lenhard, Segerer, \& Suggate, 2015). Additionally, standardized cognitive ability tests frequently incorporate direct language measures (e.g., Bayley, 2006; Gershon et al., 2013; Wechsler \& Kodama, 1949).

Parent report measures are widely utilized in psychological research. They are particularly popular as screening methods to identify developmental delays (Diamond \& Squires, 1993; Pontoppidan, Niss, Pejtersen, Julian, \& Væver, 2017). However, it is important to acknowledge that parent reports come with certain caveats, including the potential for selective reporting and social desirability bias. As a consequence, providing a comprehensive assessment of the overall quality and usefulness of these measures is challenging (Morsbach \& Prinz, 2006). Nonetheless, some parent report measures have been found to be both reliable and valid (Bodnarchuk \& Eaton, 2004; Hornman, Kerstjens, Winter, Bos, \& Reijneveld, 2013; Ireton \& Glascoe, 1995; Macy, 2012; Saudino et al., 1998).

In child language research, parent report measures are often utilized with very young children when direct assessment is challenging. One widely used measure is the MacArthur-Bates Communicative Development Inventories (CDI) (Fenson et al., 2007). The CDI asks parents to to check those words from a checklist that they believe their child produces and/or understands. This measure has been adapted for a wide range spoken and signed languages (see Frank, Braginsky, Yurovsky, \& Marchman, 2021 for an overview), with various versions available (e.g., Makransky, Dale, Havmose, \& Bleses, 2016; Mayor \& Mani, 2019), including an online version (DeMayo et al., 2021). Collaborative efforts have facilitated the pooling of data from thousands of children learning different languages into centralized repositories (Frank, Braginsky, Yurovsky, \& Marchman, 2017; Jørgensen, Dale, Bleses, \& Fenson, 2010). Importantly, the CDI exhibits validity as parental reports align with direct observations and assessments of child language (Bornstein \& Haynes, 1998; Dale, 1991; Feldman et al., 2005; Fenson et al., 1994).

However, the use of the CDI -- in typically developing children -- is limited to 37 months of age. Consequently, there is a need for a comparable measure that can be applied to older children, as parental reports offer a convenient and comprehensive means of assessing children's language abilities and provide a complementary perspective on developent. Existing instruments focusing on general cognitive development often include language scales; however, these scales lack detailed information and fail to capture individual differences effectively (Ireton \& Glascoe, 1995). For example, the Ages and Stages Questionnaire (ASQ) at 36 months comprises only six items that encompass general communicative behavior, such as whether the child can say their full name when prompted (Squires, Bricker, Twombly, et al., 2009). One notable example of a dedicated language measure for older children is the Developmental Vocabulary Assessment for Parents (DVAP, Libertus, Odic, Feigenson, \& Halberda, 2015). The DVAP is derived from the words used in the Peabody Picture Vocabulary Test {[}PPVT; Dunn and Dunn (1965){]}, a widely used direct measure of receptive vocabulary. As perhaps expected, the DVAP demonstrates high validity, as evidenced by its strong correlation with the PPVT. However, the proprietary nature of the PPVT limits the utility of the DVAP for researchers.\footnote{When the first author approached the license holder of the PPVT in Germany to ask if we could use the words to build a parental report measure, we were told that we would have to pay for every administration of the new measure and we would not be allowed to openly share the materials.} As a consequence, it is unlikely that comparable ``success story'' -- as observed with the CDI -- will emerge where researchers have adapted the original English form to different languages and more efficient forms.

More general issue with langueage measures is lack of psychometric considerations during development. The developmental process is not transparent IRT provides a toolkit that is very useful for development. Each item is assessed in its usefulness to measure a latent construct at different ages

\hypertarget{the-current-study}{%
\section{The current study}\label{the-current-study}}

Our goals was to develop a high-quality and easy-access vocabulary checklist for children between three and eight years of age. The measure and all associated materials should be openly available for other researchers to use. To ensure the quality of the items, we created a large initial item pool from which we selected high-quality items. To ensure easy-access, we implemented the checklist as an interactive web-app.

\hypertarget{task-design-an-implementation}{%
\section{Task design an implementation}\label{task-design-an-implementation}}

\hypertarget{item-pool-generation}{%
\section{Item pool generation}\label{item-pool-generation}}

\hypertarget{item-selection}{%
\section{Item selection}\label{item-selection}}

\hypertarget{participants}{%
\subsection{Participants}\label{participants}}

\hypertarget{descriptive-results}{%
\subsection{Descriptive results}\label{descriptive-results}}

\hypertarget{automated-item-selection}{%
\subsection{Automated item selection}\label{automated-item-selection}}

\hypertarget{differential-item-functioning}{%
\subsection{Differential item functioning}\label{differential-item-functioning}}

\hypertarget{psychometric-properties-of-new-checklist}{%
\section{Psychometric properties of new checklist}\label{psychometric-properties-of-new-checklist}}

\hypertarget{reliability}{%
\subsection{Reliability}\label{reliability}}

\hypertarget{convergent-validity}{%
\subsection{Convergent validity}\label{convergent-validity}}

\hypertarget{discussion}{%
\section{Discussion}\label{discussion}}

Measure is likely different in type from CDI. Especially early in life, CDI likely captures the entire vocabulary.

\hypertarget{limitations}{%
\subsection{Limitations}\label{limitations}}

Not a representative sample

\hypertarget{conclusion}{%
\subsection{Conclusion}\label{conclusion}}

\hypertarget{open-practices-statement}{%
\section{Open Practices Statement}\label{open-practices-statement}}

The task can be accessed via the following website: \url{https://ccp-odc.eva.mpg.de/orev-demo/}. The corresponding source code can be found in the following repository: \url{https://github.com/ccp-eva/orev-demo}. The data sets generated during and/or analysed during the current study are available in the following repository: \url{https://github.com/ccp-eva/orev/}. Data collection was preregistered at: \url{https://osf.io/qzstk}.

\newpage

\hypertarget{references}{%
\section{References}\label{references}}

\hypertarget{refs}{}
\begin{CSLReferences}{1}{0}
\leavevmode\vadjust pre{\hypertarget{ref-bayley2006bayley}{}}%
Bayley, N. (2006). \emph{Bayley scales of infant and toddler development--third edition}. San Antonio, TX: Harcourt Assessment.

\leavevmode\vadjust pre{\hypertarget{ref-bleses2016early}{}}%
Bleses, D., Makransky, G., Dale, P. S., Højen, A., \& Ari, B. A. (2016). Early productive vocabulary predicts academic achievement 10 years later. \emph{Applied Psycholinguistics}, \emph{37}(6), 1461--1476.

\leavevmode\vadjust pre{\hypertarget{ref-bodnarchuk2004can}{}}%
Bodnarchuk, J. L., \& Eaton, W. O. (2004). Can parent reports be trusted?: Validity of daily checklists of gross motor milestone attainment. \emph{Journal of Applied Developmental Psychology}, \emph{25}(4), 481--490.

\leavevmode\vadjust pre{\hypertarget{ref-bornstein2018stability}{}}%
Bornstein, M. H., Hahn, C.-S., Putnick, D. L., \& Pearson, R. M. (2018). Stability of core language skill from infancy to adolescence in typical and atypical development. \emph{Science Advances}, \emph{4}(11), eaat7422.

\leavevmode\vadjust pre{\hypertarget{ref-bornstein1998vocabulary}{}}%
Bornstein, M. H., \& Haynes, O. M. (1998). Vocabulary competence in early childhood: Measurement, latent construct, and predictive validity. \emph{Child Development}, \emph{69}(3), 654--671.

\leavevmode\vadjust pre{\hypertarget{ref-dale1991validity}{}}%
Dale, P. S. (1991). The validity of a parent report measure of vocabulary and syntax at 24 months. \emph{Journal of Speech, Language, and Hearing Research}, \emph{34}(3), 565--571.

\leavevmode\vadjust pre{\hypertarget{ref-demayo2021web}{}}%
DeMayo, B., Kellier, D., Braginsky, M., Bergmann, C., Hendriks, C., Rowland, C. F., \ldots{} Marchman, V. (2021). Web-CDI: A system for online administration of the MacArthur-bates communicative development inventories. \emph{Language Development Research}.

\leavevmode\vadjust pre{\hypertarget{ref-diamond1993role}{}}%
Diamond, K. E., \& Squires, J. (1993). The role of parental report in the screening and assessment of young children. \emph{Journal of Early Intervention}, \emph{17}(2), 107--115.

\leavevmode\vadjust pre{\hypertarget{ref-dunn1965peabody}{}}%
Dunn, L. M., \& Dunn, L. M. (1965). \emph{Peabody picture vocabulary test}.

\leavevmode\vadjust pre{\hypertarget{ref-dunn1997british}{}}%
Dunn, L. M., Dunn, L. M., Whetton, C., \& Burley, J. (1997). British picture vocabulary scale 2nd edition (BPVS-II). \emph{Windsor, Berks: NFER-Nelson}.

\leavevmode\vadjust pre{\hypertarget{ref-feldman2005concurrent}{}}%
Feldman, H. M., Dale, P. S., Campbell, T. F., Colborn, D. K., Kurs-Lasky, M., Rockette, H. E., \& Paradise, J. L. (2005). Concurrent and predictive validity of parent reports of child language at ages 2 and 3 years. \emph{Child Development}, \emph{76}(4), 856--868.

\leavevmode\vadjust pre{\hypertarget{ref-fenson2007macarthur}{}}%
Fenson, L. et al. (2007). \emph{MacArthur-bates communicative development inventories}. Paul H. Brookes Publishing Company Baltimore, MD.

\leavevmode\vadjust pre{\hypertarget{ref-fenson1994variability}{}}%
Fenson, L., Dale, P. S., Reznick, J. S., Bates, E., Thal, D. J., Pethick, S. J., \ldots{} Stiles, J. (1994). Variability in early communicative development. \emph{Monographs of the Society for Research in Child Development}, i--185.

\leavevmode\vadjust pre{\hypertarget{ref-frank2017wordbank}{}}%
Frank, M. C., Braginsky, M., Yurovsky, D., \& Marchman, V. A. (2017). Wordbank: An open repository for developmental vocabulary data. \emph{Journal of Child Language}, \emph{44}(3), 677--694.

\leavevmode\vadjust pre{\hypertarget{ref-frank2021variability}{}}%
Frank, M. C., Braginsky, M., Yurovsky, D., \& Marchman, V. A. (2021). \emph{Variability and consistency in early language learning: The wordbank project}. MIT Press.

\leavevmode\vadjust pre{\hypertarget{ref-gershon2013iv}{}}%
Gershon, R. C., Slotkin, J., Manly, J. J., Blitz, D. L., Beaumont, J. L., Schnipke, D., et al.others. (2013). IV. NIH toolbox cognition battery (CB): Measuring language (vocabulary comprehension and reading decoding). \emph{Monographs of the Society for Research in Child Development}, \emph{78}(4), 49--69.

\leavevmode\vadjust pre{\hypertarget{ref-gluck2011wortschatz}{}}%
Glück, C. W., \& Glück, C. W. (2011). \emph{Wortschatz-und wortfindungstest f{ü}r 6-bis 10-j{ä}hrige (WWT 6-10)}. Urban \& Fischer.

\leavevmode\vadjust pre{\hypertarget{ref-golinkoff2017user}{}}%
Golinkoff, Roberta M., De Villiers, J. G., Hirsh-Pasek, K., Iglesias, A., Wilson, M. S., Morini, G., \& Brezack, N. (2017). \emph{User's manual for the quick interactive language screener (QUILS): A measure of vocabulary, syntax, and language acquisition skills in young children}. Paul H. Brookes Publishing Company.

\leavevmode\vadjust pre{\hypertarget{ref-golinkoff2019language}{}}%
Golinkoff, Roberta Michnick, Hoff, E., Rowe, M. L., Tamis-LeMonda, C. S., \& Hirsh-Pasek, K. (2019). Language matters: Denying the existence of the 30-million-word gap has serious consequences. \emph{Child Development}, \emph{90}(3), 985--992.

\leavevmode\vadjust pre{\hypertarget{ref-hornman2013validity}{}}%
Hornman, J., Kerstjens, J. M., Winter, A. F. de, Bos, A. F., \& Reijneveld, S. A. (2013). Validity and internal consistency of the ages and stages questionnaire 60-month version and the effect of three scoring methods. \emph{Early Human Development}, \emph{89}(12), 1011--1015.

\leavevmode\vadjust pre{\hypertarget{ref-ireton1995assessin}{}}%
Ireton, H., \& Glascoe, F. P. (1995). Assessin children's development using parents' reports: The child development inventory. \emph{Clinical Pediatrics}, \emph{34}(5), 248--255.

\leavevmode\vadjust pre{\hypertarget{ref-jorgensen2010clex}{}}%
Jørgensen, R. N., Dale, P. S., Bleses, D., \& Fenson, L. (2010). CLEX: A cross-linguistic lexical norms database. \emph{Journal of Child Language}, \emph{37}(2), 419--428.

\leavevmode\vadjust pre{\hypertarget{ref-kauschke2002patholinguistische}{}}%
Kauschke, C., \& Siegmüller, J. (2002). \emph{Patholinguistische diagnostik bei sprachentwicklungsst{ö}rungen: Diagnostikband phonologie}. Elsevier Urban \& Fischer.

\leavevmode\vadjust pre{\hypertarget{ref-kiese2005awst}{}}%
Kiese-Himmel, C. (2005). AWST-r-aktiver wortschatztest f{ü}r 3-bis 5-j{ä}hrige kinder (AWST-r--active vocabulary test for 3-to 5-year-old children). \emph{G{ö}ttingen: Hogrefe}.

\leavevmode\vadjust pre{\hypertarget{ref-lenhard2015peabody}{}}%
Lenhard, A., Lenhard, W., Segerer, R., \& Suggate, S. (2015). \emph{Peabody picture vocabulary test-4. Ausgabe: Deutsche fassung}. Frankfurt am Main: Pearson Assessment.

\leavevmode\vadjust pre{\hypertarget{ref-libertus2015developmental}{}}%
Libertus, M. E., Odic, D., Feigenson, L., \& Halberda, J. (2015). A developmental vocabulary assessment for parents (DVAP): Validating parental report of vocabulary size in 2-to 7-year-old children. \emph{Journal of Cognition and Development}, \emph{16}(3), 442--454.

\leavevmode\vadjust pre{\hypertarget{ref-macy2012evidence}{}}%
Macy, M. (2012). The evidence behind developmental screening instruments. \emph{Infants \& Young Children}, \emph{25}(1), 19--61.

\leavevmode\vadjust pre{\hypertarget{ref-makransky2016item}{}}%
Makransky, G., Dale, P. S., Havmose, P., \& Bleses, D. (2016). An item response theory--based, computerized adaptive testing version of the MacArthur--bates communicative development inventory: Words \& sentences (CDI: WS). \emph{Journal of Speech, Language, and Hearing Research}, \emph{59}(2), 281--289.

\leavevmode\vadjust pre{\hypertarget{ref-marchman2008speed}{}}%
Marchman, V. A., \& Fernald, A. (2008). Speed of word recognition and vocabulary knowledge in infancy predict cognitive and language outcomes in later childhood. \emph{Developmental Science}, \emph{11}(3), F9--F16.

\leavevmode\vadjust pre{\hypertarget{ref-mayor2019short}{}}%
Mayor, J., \& Mani, N. (2019). A short version of the MacArthur--bates communicative development inventories with high validity. \emph{Behavior Research Methods}, \emph{51}(5), 2248--2255.

\leavevmode\vadjust pre{\hypertarget{ref-morgan201524}{}}%
Morgan, P. L., Farkas, G., Hillemeier, M. M., Hammer, C. S., \& Maczuga, S. (2015). 24-month-old children with larger oral vocabularies display greater academic and behavioral functioning at kindergarten entry. \emph{Child Development}, \emph{86}(5), 1351--1370.

\leavevmode\vadjust pre{\hypertarget{ref-morsbach2006understanding}{}}%
Morsbach, S. K., \& Prinz, R. J. (2006). Understanding and improving the validity of self-report of parenting. \emph{Clinical Child and Family Psychology Review}, \emph{9}, 1--21.

\leavevmode\vadjust pre{\hypertarget{ref-pace2019measuring}{}}%
Pace, A., Alper, R., Burchinal, M. R., Golinkoff, R. M., \& Hirsh-Pasek, K. (2019). Measuring success: Within and cross-domain predictors of academic and social trajectories in elementary school. \emph{Early Childhood Research Quarterly}, \emph{46}, 112--125.

\leavevmode\vadjust pre{\hypertarget{ref-pace2017identifying}{}}%
Pace, A., Luo, R., Hirsh-Pasek, K., \& Golinkoff, R. M. (2017). Identifying pathways between socioeconomic status and language development. \emph{Annual Review of Linguistics}, \emph{3}, 285--308.

\leavevmode\vadjust pre{\hypertarget{ref-pontoppidan2017parent}{}}%
Pontoppidan, M., Niss, N. K., Pejtersen, J. H., Julian, M. M., \& Væver, M. S. (2017). Parent report measures of infant and toddler social-emotional development: A systematic review. \emph{Family Practice}, \emph{34}(2), 127--137.

\leavevmode\vadjust pre{\hypertarget{ref-saudino1998validity}{}}%
Saudino, K. J., Dale, P. S., Oliver, B., Petrill, S. A., Richardson, V., Rutter, M., \ldots{} Plomin, R. (1998). The validity of parent-based assessment of the cognitive abilities of 2-year-olds. \emph{British Journal of Developmental Psychology}, \emph{16}(3), 349--362.

\leavevmode\vadjust pre{\hypertarget{ref-schoon2010children}{}}%
Schoon, I., Parsons, S., Rush, R., \& Law, J. (2010). Children's language ability and psychosocial development: A 29-year follow-up study. \emph{Pediatrics}, \emph{126}(1), e73--e80.

\leavevmode\vadjust pre{\hypertarget{ref-squires2009ages}{}}%
Squires, J., Bricker, D. D., Twombly, E., et al. (2009). \emph{Ages \& stages questionnaires}. Paul H. Brookes Baltimore, MD.

\leavevmode\vadjust pre{\hypertarget{ref-walker1994prediction}{}}%
Walker, D., Greenwood, C., Hart, B., \& Carta, J. (1994). Prediction of school outcomes based on early language production and socioeconomic factors. \emph{Child Development}, \emph{65}(2), 606--621.

\leavevmode\vadjust pre{\hypertarget{ref-wechsler1949wechsler}{}}%
Wechsler, D., \& Kodama, H. (1949). \emph{Wechsler intelligence scale for children} (Vol. 1). Psychological corporation New York.

\end{CSLReferences}


\end{document}
